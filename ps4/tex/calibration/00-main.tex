\item \points{8} {\bf Model Calibration.}
\renewcommand{\Pr}{P}
In this question, we will discuss the uncertainty qualification of the machine learning models, and, in particular, a specific notion of uncertainty quantification called calibration. 
In many  real-world applications such as policy-making or health care, prediction by machine learning models is not the end goal. Such predictions are used as  the input to decision making process which needs to balances risks and rewards, both of which are often uncertain. Therefore, machine learning models not only need to produce a single prediction for each example, but also to measure the uncertainty of the prediction. 

In this question we consider the binary classification setting. Let $\mathcal{X}$ be the input space and $\mathcal{Y}=\{0,1\}$ be the label space. For simplicity, we assume that $\mathcal{X}$ is a discrete space of finite number of elements. Let $X$ and $Y$ be random variables denoting the input and label, respectively, over $\mathcal{X}$ and $\mathcal{Y}$, respectively. Let $X,Y$ have a joint distribution $\mathcal{P}$. 
Suppose we have a model $h:\mathcal{X} \rightarrow [0,1]$ (here we dropped the dependency on the parameters $\theta$ because the parameterization is not relevant to the discussion.) Recall that in logistic regression, we assume that there exists some model $h^\star(\cdot)$ (parameterized by a certain form, e.g., linear functions or neural networks) such that the output of the model $h^\star$ represents the conditional probability of $Y$ being 1 given $X$:
\begin{align}
P\left[Y = 1 \vert X = x\right] = h^\star(x) \label{eqn:1}
\end{align}
%the  output of the model represents the model’s confidence (probability) that the label is $1$.  
Under this assumption, we can derive the logistic loss to train and obtain some model $h(\cdot)$. Recall that the decision boundary is typically set to correspond to $h(x) = 1/2$, which means that the prediction of $x$ is 1 if $h(x) \ge 1/2$, and 0 if $h(x) < 1/2$. To quantify the uncertainty of the prediction, it's tempting to just use the value of $h(x)$ --- the closer $h(x)$ is to 0 or 1, the more confident we are with our prediction. 

How do we know whether the learned model $h(\cdot)$ indeed outputs a reliable and truthful probability $h(x)$? We note that we shouldn't take it for granted because a) the assumption~\eqref{eqn:1} may not exactly be satisfied by the data, and b) even if the assumption~\eqref{eqn:1} holds perfectly, the learned model $h(x)$ may be different from the true model $h^\star$ that satisfies~\eqref{eqn:1}. 

We will introduce a  metric to evaluate how reliably the probabilities output by a model $h$ capture the confidence of the model.
In order for these probabilities to be useful as confidence measures, we would ideally like them to be \emph{calibrated} in the following sense.  
Calibration intuitively requires that among all the examples for which the model predicts the value 0.7, indeed 70\% of them should have label 1. %that whenever a model assigns $0.7$ probability to an event, it should be the case that the event actually holds about $70\%$ of the time. 


Formally, we say that a model $h$ is perfectly calibrated if for all possible probabilities $p \in [0,1]$ such that $\Pr[h(X)= p ] > 0$, \begin{align}
\Pr[Y= 1\mid h(X) =p] =p.\label{eqn:4}\end{align} Recall that $(X,Y)$ is a random draw from the (population) data distribution. 

In the example in Table~\ref{tab:t-example}, the model $h$ is not perfectly calibrated, because when $p = 0.3$, $\Pr[Y= 1\mid h(X) =p] \neq p$, 


\begin{align}
P[Y= 1\mid h(X) = 0.3] & = \frac{\Pr[Y = 1 \textup{ and } h(X) = 0.3]}{\Pr[h(X) = 0.3]} \nonumber\\
& = \frac{\Pr[Y= 1 \textup{ and } (X = 0 \textup{ or } 1) ]}{\Pr[X = 0 \textup{ or } 1]} \nonumber\\
& = \frac{\Pr[Y= 1 \textup{ and } X = 0 ] + \Pr[Y= 1 \textup{ and } X = 1 ]}{P[X = 0 \textup{ or } 1]} \nonumber\\
& = \frac{\Pr[Y= 1 \mid  X = 0 ]\Pr[X=0] + \Pr[Y= 1 \mid  X = 1]\Pr[X=1]}{P[X = 0 \textup{ or } 1]} \nonumber\\
& = \frac{0.2\times 0.25+ 0.0\times 0.25}{0.5} = 0.1 \neq 0.3 \nonumber
\end{align}



\DeclarePairedDelimiterX{\infdivx}[2]{(}{)}{%
	#1\;\delimsize\|\;#2%
}
\newcommand{\infdiv}{D_{\text{KL}}\infdivx}


\begin{enumerate}

	
\item \subquestionpoints{3}
%Assume we have a binary classification model that is perfectly calibrated.
Show that
the perfect calibration does not necessarily imply that the model achieves perfect accuracy. Is the
converse necessarily true? Justify your answers by providing either a proof or a counterexample. (Note that perfect accuracy means that $\Pr[\mathbb{I} [h(X) \ge 0.5] = Y] = 1$ \footnote{$\mathbb{I}$ is the indicator function. $\mathbb{I}[h(x) \ge 0.5]$ is equal to $1$ if $h(x) \ge 0.5$ and is equal to $0$ if $h(x) < 0.5$.}.)



	
	\ifnum\solutions=1 {
		\begin{answer}
\end{answer}

	} \fi

	\item[(b)] \subquestionpoints{5}
As you showed in the last part, calibration  by  itself  does  not necessarily guarantee good accuracy. 
Good models must also be sharp, i.e., the probabilities output by the model should be close to $0$ or $1$. 
Mean squared error (MSE) is a common measure for evaluating the quality of a model.
\begin{align}
	\text{MSE}(h) = \E \left[(Y-h(X))^2\right]
\end{align}

\begin{table}
	\centering
	\begin{tabular}{ccccc} \toprule
		$x$&$P[X=x]$&$P[Y=1 \mid X=x]$&$h(x)$&$T(x)=\E [Y \mid h(X)=h(x)]$\\ \midrule
		0&0.25&0.2&0.3&0.1\\
		1&0.25&0.0&0.3&0.1\\
		2&0.25&1.0&0.9&0.9\\ 
		3&0.25&0.8&0.9&0.9\\ \bottomrule
	\end{tabular}
	\caption{\label{tab:t-example} An example of a model $h : \mathcal{X}=\{0,1,2,3\} \rightarrow [0,1]$.  
		For calculating $T(x)$, we look at all data points with the same score $h(x)$, and compute the probability of $Y=1$ for these data points.}
\end{table}



In this part, we will show that the MSE can be decomposed to two parts, such that one part corresponds to the calibration error, and the other part corresponds to the sharpness of the model.


Formally, let $T(x) = \Pr [Y =1 \mid h(X) = h(x)]$ denote the true probability of $Y=1$ given that the prediction is equal to $h(x)$.
Intuitively, for a data point $x$, $T(x)$ is the probability of $Y=1$ for all the data points with the same score as $x$. See Table~\ref{tab:t-example} for an example.

Define calibration error CE to be\footnote{There are other definition of calibration errors, e.g., the one introduced in the next part.}:
\begin{align}
	\text{CE}(h) = \E [(T(X) - h(X))^2]
\end{align}
The calibration error here is a quantitative instantiation of the notion of perfect calibration in equation~\eqref{eqn:4}.  Indeed,  zero calibration error implies perfect calibration: zero calibration means the model perfectly predicts the true probability, i.e., $h(X) = T(X)$ w.p. 1. This in turns implies that the model is perfectly calibrated because for any $p$ such that $\Pr[h(X)=p] > 0$, we can take some $x_0$ such that $h(x_0) = p$ and $h(x_0)=T(x_0)$, and conclude
\begin{align}
P[Y=1\mid h(X)= p] & = P[Y=1\mid h(X)= h(x_0)] \\
& = T(x_0) = h(x_0) \tag{by calibration error = 0} \\
& = p \tag{by the assumption that $h(x_0)= p$}
\end{align}
As explained before, we want our model prediction to be sharp as well. 
One way to define sharpness of a model is to look at the variance of $T(X)$. 
Let's define sharpness of model $h$ as follows:
\begin{align}
\text {SH}(h) = \text {Var} (T(X))
\end{align}
The sharpness term measures how much variation there is in the true probability across model prediction.
It is small when $T(x)$ is similar across all data points. To get a better intuition for sharpness, let's group data points according to the model prediction; i.e., all data points with the same prediction ($h(X)$) are in the same group. 
In each group we want the true underlying probability of labels $T(X)=\mathbb{E}[Y | h(X)]$ to be close to $h(X)$ (calibration). 
For sharpness, we want the value of $T(X) = \mathbb{E} [Y | h(X)]$ to be spread out, i.e., not all the groups have similar values. 
$\text {Var} [T(X)]$ is one way to measure how spread out $T(X)$ is.

MSE can be decomposed as follows:
\begin{align}
	\label{eqn:mse-decomposiiton}
	\text {MSE}(h) = \underbrace{\text {Var}[Y]}_{\text {Instrinsic uncertainty}} - \underbrace{\text {Var} [T(X)]}_\text{Sharpness} + \underbrace{\E \left [(T(X) - h(X))^2\right]}_\text{Calibration error}
\end{align}

This decomposition states that by minimizing MSE we try to find a sharp model with small calibration error.
In other words, when we choose a model with minimum MSE it means between two models with the same calibration error we prefer the one that is sharper and between two models with the same sharpness we prefer the one with lower calibration error.
Note that the uncertainty term does not depend on the model and can be mostly ignored.

\noindent{\bf Prove that the decomposition in Equation~\eqref{eqn:mse-decomposiiton} is correct.}

\paragraph{Hint.}
Recall that using the law of total variance, we can decompose the variance of $Y$ onto $h(X)$:
$$	\text {Var}[Y] = \E [\text {Var}[Y \mid h(X)]] + \text {Var} [\E [Y \mid h(X)]]$$

	
	\ifnum\solutions=1 {
		\begin{answer}
\end{answer}
	} \fi
\end{enumerate}

\begin{enumerate}
	\item[(c)] \subquestionpoints{0} [{\bf{This part is optional and does not have any points}}]
In the previous part, we studied MSE and decomposed it to two terms corresponding to the sharpness and calibration error.
But as we explained, there are other different ways to measure the calibration and sharpness of a model.
In this part we focus on logistic regression models.
In particular, we are going to show logistic loss can also be decomposed to two terms; where one term can be interpreted as the sharpness of the prediction (which we call log-sharpness) and the other can be interpreted as the calibration error (which we call log-calibration-error).  
Recall that the logistic loss (on population) of the model $h$ is defined as:
\begin{align}
\text {Log-Loss}(h) = \E [-Y \log (h(X)) - (1-Y) \log (1-h(X))]
\end{align}

Prove that logistic loss can be decomposed to two terms as follows:
\begin{align}
\text {Log-Loss}(h) =&\underbrace{\E \left [T(X) \log \left(\frac{T(X)}{h(X)}\right) + (1-T(X)) \log \left(\frac{1-T(X)}{1-h(X)}\right)\right ]}_\text {log-calibration-error} \nonumber \\
&-\underbrace{\E \left [T(X) \log (T(X)) + (1-T(X)) \log (1-T(X)) \right]}_\text {log-sharpness}\label{eqn:log-decompose}
\end{align}

Discuss why the log-calibration-error term in \eqref{eqn:log-decompose} is a meaningful term for measuring the calibration error and specify when it attains its minimum.
Similarly discuss why log-sharpness term in \eqref{eqn:log-decompose}  is a meaningful term for measuring the sharpness of a model and specify when it attains its maximum.

\paragraph{Hint.} 

For showing that log-calibration-error and log-sharpness are meaningful terms, you should use your information theory knowledge (there is a section about information theory in question 4).
%
For each data point $x$, both model prediction ($h(x)$) and underlying probability ($T(x)$) define a distribution over the label set $\mathcal{Y} = \{0,1\}$.
In particular, define distribution $P_1$ on $\mathcal{Y}=\{0,1\}$ as follows:
$P_1(Y=1) = T(x)$ and $P_1(Y=0) = 1-T(x)$. 
Similarly, define distribution $P_2$ on $\mathcal{Y}=\{0,1\}$ as follows: $P_2(Y=1) = h(x)$ and $P_2(Y=0) = 1-h(x)$. 
You can interpret the log-calibration-error in \eqref{eqn:log-decompose} as KL-divergence distance between these two distributions. 
Recall that the KL divergence distance between these two distributions is:
\begin{align}
	\infdiv{P_1}{P_2} = P_1(Y=0) \log\left(\frac{P_1(Y=0)}{P_2(Y=0)}\right) + P_1(Y=1) \log\left(\frac{P_1(Y=1)}{P_2(Y=1)}\right) 
\end{align} 
The log-sharpness term in \eqref{eqn:log-decompose} can be expressed as the negative entropy of the distribution corresponding to $T(x)$. 
Recall that entropy of distribution $P_1$ is:
\begin{align}
	\text {H}(P_1) = -P_1(Y=0)\log(P_1(Y=0)) - P_1(Y=1)\log(P_1(Y=1))
\end{align}

\noindent{\bf Remark: } 
The decomposition suggests that minimizing the logistic loss (on the population) tends to minimize the calibration error as well, since the calibration error is upper bounded by the logistic loss. In practice, when the train and test sets are from the same distribution and when the model has not overfit or underfit, logistic regression tends to be well calibrated on the test data as well.  In contrast, modern large-scale deep learning models trained with the logistic loss are typically not well-calibrated, likely since the population  logistic loss suffers from overfitting (the test loss is much higher than the train loss), even when there is little overfitting in terms of the accuracy. As such, often people use other recalibration methods to adjust the outputs of a deep learning model to be better calibrated.

	
	\ifnum\solutions=1 {
		\begin{answer}
\end{answer}
	} \fi
\end{enumerate}
